\subsection{Key Management and Wallet Security}
Proper key management is the most important part of security for any blockchain-based system. Even the strongest protocols can fail if keys are not handled correctly \cite{fumy1993}. In decentralized systems, private keys give users final control over their digital assets, identity, and ability to perform actions on the blockchain. This idea is often summarized by the saying, "Not your keys, not your coins," but it applies to more than just currency \cite{yu2024}. The main tools users have for managing these keys are called "wallets." The security of these wallets is therefore essential for protecting user actions on the blockchain \cite{suratkar2020}. However, wallet security is not just a technical problem; it also depends on software design and, importantly, on the behavior and understanding of the users themselves \cite{yu2024}.

To understand wallet security, it is helpful to first classify the different types of wallets. The most basic classification is between "hot wallets," which are connected to the internet, and "cold wallets," which are kept offline. Hot wallets include desktop software, mobile apps, and web browser extensions. They are easier to use for daily transactions, but their online nature makes them more vulnerable to attacks. Cold wallets, such as hardware devices or paper wallets, offer better security for long-term storage because they are not directly exposed to online threats \cite{suratkar2020}. Another important classification is based on who controls the keys. With "custodial wallets," a third party like a cryptocurrency exchange holds the keys for the user. This is simpler for beginners, as the experience is similar to online banking, but it requires trusting that the third party is competent and honest. With "non-custodial wallets," users have full control and responsibility over their own keys. These non-custodial wallets can be further divided into traditional Externally Owned Accounts (EOAs) and newer Smart Contract wallets, which allow for more complex security rules \cite{yu2024, suratkar2020}.

The vulnerabilities in these systems exist at multiple levels, but the most common threats are those on the user's own device \cite{houy2023}. A major technical risk is the improper storage of keys, such as saving them as unencrypted plaintext in the device's memory, where they can be stolen by malware. Flaws in the wallet software, such as insecure interfaces or the use of buggy code libraries, also create significant risks. This is not just a theoretical problem; real-world attacks often exploit these weaknesses. For example, weak key generation methods like "brain wallets," which use simple, memorable phrases, are a critical vulnerability. The low entropy of human-generated phrases makes them easy to guess, and one study found that most such wallets were drained of funds in less than 24 hours \cite{houy2023}. At the institutional level, the history of exchange hacks like the infamous Mt. Gox incident shows that even large platforms can have critical flaws \cite{houy2023}. More recently, the collapse of the FTX exchange served as a powerful reminder of counterparty risk—the danger that the trusted third party will fail due to mismanagement or fraud, leading to a total loss of user funds \cite{yu2024}. Even at an individual level, social threats are a major risk; one study documented a user who lost all their funds simply because they let a friend see their login details during the wallet setup process, highlighting the dangers of misplaced trust \cite{yu2024}.

To protect against these threats, both technical and user-driven defense methods are used. The main technical defense is to use cold storage, such as hardware wallets, to keep keys offline and safe from online hackers \cite{suratkar2020}. More advanced solutions include multi-signature and smart contract wallets, which allow for programmable security rules like spending limits or requiring multiple people to approve a transaction \cite{yu2024}. However, since many attacks target the user, user-driven strategies are just as important. A common and effective strategy is "risk diversification," where users spread their assets across multiple wallets. For instance, a user might keep a small amount of "spending money" in a convenient mobile hot wallet, while keeping the majority of their savings in a more secure cold wallet \cite{yu2024}. They also use different wallets for different tasks, for example, using a dedicated wallet with minimal funds for interacting with new or risky dApps. For high-value transactions, many users prefer a PC setup because they can use third-party security extensions, like Fire or Revoke.cash, which simulate transactions and warn them about malicious smart contracts before they sign \cite{yu2024}.

We can see these security trade-offs in the real-world systems that users choose. Centralized exchanges like Coinbase offer a simple user experience that is similar to online banking. This makes them popular with beginners, but it comes with significant counterparty risk, as tragically demonstrated by the failure of FTX \cite{yu2024}. Hardware wallets like Ledger or Trezor represent the opposite approach. They provide high security by giving users full control over their offline keys, but they can be difficult to use and require the user to be fully responsible for their own security. This trust model has also been challenged recently. For example, Ledger's controversial "Recover" service, which proposed storing shards of a user's seed phrase with third parties, caused a backlash because it went against the core reason users chose a hardware wallet: to be the sole holder of their keys \cite{yu2024}. As a middle ground, new systems like smart contract wallets (e.g., Argent) are emerging. They try to offer the best of both worlds: strong security features like social recovery to prevent key loss, combined with an easier user experience that often removes the need to manually manage a seed phrase. These different models show that the market is still searching for the right balance between security, usability, and trust \cite{yu2024}.