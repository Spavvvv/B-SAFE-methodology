\subsection{DeFi Protocol Risks}
Decentralized Financial ecosystem (DeFi), is built based on blockchain platforms such as Ethereum, has emerged as an alternative to Centralized Finance due to its transparency, traceability, and decentralized nature. DeFi offers a wide range of financial services, primarily implemented through smart contracts. However, the rapid growth of DeFi has also come with serious security risks, leading to significant financial losses. While blockchain technology itself is considered secure due to its properties such as immutability and consensus mechanisms, the applications and additional layers built on top of blockchain – namely DeFi protocols – are not entirely secure and can be vulnerable.

Many recent works have systematized DeFi into layers (network, consensus, smart-contract, protocol, auxiliary services) and emphasized that many incidents arise from unsafe dependencies between protocols and off-chain services (oracles, centralized relays, bridges) \cite{zhou2023sok}. Among them, vulnerabilities in the DeFi protocol layer (PRO Layer) are often related to design flaws or financial market manipulation. For instance, pricing mechanisms, slippage, liquidation mechanisms, rebases… or invalid assumptions about token standards can be catastrophic when contracts are composed together; in particular, external dependencies are called directly without consistency checks are the source of many real-world failures. \cite{zhou2023sok}

A key economic risk is flash loans, uncollateralized lending mechanisms in an atomic transaction. Flash loans have opened a new attack vector where an attacker can temporarily borrow large amounts of capital to manipulate the market or price feed, performing a series of profit and debt repayment operations in the same transaction. Attacks like Harvest, PancakeBunny, Beanstalk… \cite{zhou2023sok, li2022securitydefi} show that flash loans lower the cost barrier to attack and make small design issues become financial catastrophic. Another risk directly related to off-chain backends is that when price data sources are manipulated – through source changes, on-chain update attacks, or updater compromises – key parameters such as liquidation prices or collateralization ratios can become distorted, leading to mass liquidations or systemic profiteering \cite{li2022securitydefi}. There are mitigations such as multiple source aggregation, medianizers, or latency mechanisms that exist but carry trade-offs in latency, centralization and fault tolerance \cite{zhou2023sok, li2022securitydefi}.

In addition, transaction ordering and MEV (Miner/Maximal Extractable Value) issues allow sequencers or miners to order, insert or remove transactions to maximize profits – this mechanism gives rise to front-running, sandwiching and other mining strategies, which directly impact the stability of the protocol's financial invariants \cite{zhou2023sok}. Expanding the functional space with cross-chain bridges also creates a new attack surface: many bridges rely on centralized signing/organizations, and bridge crashes have led to large scale asset losses, demonstrating a clear trade-off between cross-chain utility and security risk \cite{li2022securitydefi}. Finally, operational and human risks – including private key, mismanagement (privileged keys, weak multisig…), compromised front ends, and implement flaws (not pure protocol design flaws) have a direct impact on asset security and are often present in real-world incidents \cite{liu2024defiscams}.

To mitigate these risks, incident studies and analysis have proposed a multilayer set of measures: protocol design that considers both economic attack scenarios (game-theoretic stress testing) and defense mechanisms such as circuit breakers \cite{zhou2023sok}; oracle enhancements using aggregations, delayed updates or reputation-based models \cite{li2022securitydefi}; MEV mitigations using transparent sequencers or close-chain relay \cite{zhou2023sok}; along with audit, formal verification and real-time monitoring (e.g., oracle mutation detection) with response options as emergency halts \cite{zhou2023sok, li2022securitydefi}. Each approach carries trade-offs in performance, latency, and decentralization, so the choice of solution should be based on the specific application context.

Finally, the systematic analysis revealed important research gaps: the lack of a comprehensive quantitative framework for protocol economic risk (incorporating TVL, liquidity depth, oracle latency, and flash loan capabilities), the lack of a common fault tolerant architectural pattern for trustless backends, and the lack of dependency analysis tools for complex composability environments – these gaps share the research direction needed to improve the robustness of DeFi protocols in the broader blockchain landscape.