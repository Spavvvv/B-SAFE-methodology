\section{Smart Contracts}
	
\section{LITERATURE REVIEW}
	
\subsection{Introduction}
\textbf{Purpose:} To introduce the chapter, state its objectives, and provide a roadmap for the reader.
	
\textbf{Suggested Content:}
\begin{itemize}
	\item Begin by linking this chapter to the overall research goals stated previously.
	\item Clearly state the objective of this chapter: to build a foundational understanding of blockchain security challenges, examine the role of Machine Learning (ML) as a potential solution, and thereby identify existing research gaps.
	\item Outline the chapter's structure: ``This review is organized into three main parts: first, a survey of core security challenges in blockchain; second, an analysis of ML applications designed to address these challenges; and finally, a synthesis of findings to pinpoint opportunities for future research.''
\end{itemize}
	
\subsection{The Landscape of Blockchain Security Challenges}
\textbf{Purpose:} To establish the ``problem space'' by reviewing vulnerabilities inherent to blockchain technology, \textbf{without yet focusing on ML solutions}.
	
\subsubsection{Foundational and Protocol-Level Vulnerabilities}
\begin{itemize}
	\item \textbf{Consensus Mechanism Security:} Synthesize research on attacks against consensus protocols (e.g., 51\% attacks, Sybil attacks, Eclipse attacks).
	\item \textbf{Key Management Vulnerability:} Analyze literature on risks in cryptographic key management (e.g., insecure private key storage, brain wallet weaknesses, key leakage).
\end{itemize}
	
\subsubsection{Application-Level Vulnerabilities}
\begin{itemize}
	\item \textbf{Smart Contract Vulnerability Assessment:} Review common smart contract vulnerabilities studied in the literature (e.g., re-entrancy, integer overflow/underflow, unsafe delegate calls).
	\item \textbf{Program/Application Bugs \& Integration Risks:} Discuss research on general programming errors in wallets and DApps, and the risks associated with integrating insecure third-party libraries or protocols.
\end{itemize}
	
\subsection{Machine Learning as a Defense Mechanism}
\textbf{Purpose:} To explore the ``solution space'' by reviewing how researchers have applied ML to solve the problems identified in section 4.2.
	
\subsubsection{Applying ML to Protocol-Level Security}
\begin{itemize}
	\item Review studies that use ML for network-level threat detection, such as identifying anomalous behavior indicative of a 51\% attack or a Sybil attack.
\end{itemize}
	
\subsubsection{Applying ML to Application-Level Security}
\begin{itemize}
	\item Analyze papers that employ ML for the automated vulnerability analysis of smart contract source code or bytecode.
	\item Synthesize research using ML to detect illicit transactions, fraudulent activities, or money laundering schemes on the blockchain.
\end{itemize}
	
\subsection{Synthesis and Research Gaps}
\textbf{Purpose:} The conclusion of the literature review. This section synthesizes the findings and explicitly justifies the novelty and necessity of your research.
	
\subsubsection{Summary of the State-of-the-Art}
\begin{itemize}
	\item Provide a concise summary: ``In summary, the literature confirms significant vulnerabilities at both the protocol and application layers of blockchain. While ML has emerged as a promising approach, its application has primarily focused on smart contract analysis and anomaly detection...''
\end{itemize}
	
\subsubsection{Identifying Opportunities for Contribution}
\begin{itemize}
	\item Clearly articulate the research gaps using the ``Yes, but...'' method.
	\item \textbf{Scope Gap:} ``However, the vast majority of current research concentrates on the Ethereum blockchain, leaving a significant gap in the application of these ML techniques to other prominent platforms with different architectures, such as Solana or Polkadot.''
	\item \textbf{Methodological Gap:} ``Furthermore, most existing ML models are supervised, requiring large, labeled datasets of known attacks, which are scarce and difficult to obtain. Research into unsupervised or semi-supervised methods for detecting zero-day threats remains limited.''
	\item \textbf{Holistic Gap:} ``Finally, no unified framework currently exists that leverages ML to assess risk holistically across multiple layers, from consensus mechanisms to application-level bugs. Existing solutions tend to operate in silos.''
	\item Conclude by stating how your work addresses these gaps: ``This research, therefore, aims to address these gaps by proposing [Your Main Contribution], which will be detailed in the subsequent sections.''
\end{itemize}